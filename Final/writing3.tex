\section{Memory Management}
\subsection{Overview}
Memory management in Windows, FreeBSD, and Linux is the process of creating addresses that directly reference specific areas of a physical memory. This involves but not limited to RAM (Random Access Memory), memory caches, and flash based memory from devices like SSD's (solid state drives)\cite{Whatisme93:online}. These abstractions of physical memory are normally kept in virtual memory space as stated above in the form of addresses. For Linux this is typically done via pages and paging tables while other operating systems for example windows utilizes a slight different structured approach, leaving the overall semantics of the concept relatively the same across the board. In any of the given operating systems memory management involves the allocation, reallocation of data to specific blocks of memory while organizing it into a digestible way that both a program can quickly access but more importantly a programmer can utilize when creating user and kernel level applications.
\subsection{Memory in Windows}
Within the Windows kernel memory management is slightly different from the Linux kernel underneath the hood. The first major difference being page sizing. In Windows pages can be one of two sizes large (2MB) and small (4KB)\cite{windowsbookpt1}, mean while Linux utilizes (depending on the architecture of the system) 4KB and 8KB page sizes\cite{Linux_book}. These pages for both systems are managed in a similar way concept wise, through the use of paging directories and tables. Address mappings for memory are structured much like a tree branching from the root to different tables depending on the index of a address space to the page itself. This structure minimizes search times for a page and organizes addresses in a uniform way for kernel level developers creating drivers. While pages in Linux are the back bone of data management that can be manipulated by drivers, in Windows pages require several additional services like managers for working sets, process and stack swappers, page writers, and a thread dedicated to writing zeros to a page (because alternative methods give significant performance impacts to the Windows Kernel). Additionally pages in Windows have different states, while Linux pages have a relative first come first serve bases depending on the process allocating memory. These states in the Windows kernel are as specified: free (allocated but unused page), reserved (will be allocated and used, a preemptive approach to memory allocation for a process), committed (private to a specific process, and the page is allocated), and shareable (Non private, mapped and can be accessed by any process or thread). These states for pages in Windows help set priorities for allocation of processes.
\subsection{Memory in FreeBSD}
FreeBSD and Linux's memory management methods are extremely similar because many of its core components are exactly the same.
\begin{figure}[h]
\begin{lstlisting}[language=C, style=customc]
void *mmap(addr, len, prot, flags, fd, offset)
  void		*addr;
  size_t	len;
  int		prot;
  int		flags;
  int		fd;
  off_t		offset;
{
  if (__getosreldate() >= 700051)
    return (__sys_mmap(addr, len, prot, flags, fd, offset));
  else
    return (__sys_freebsd6_mmap(addr, len, prot, flags, fd, 0, offset));
}
\end{lstlisting}
\caption{ Memory mapping in FreeBSD\cite{mmapcinf8:online}}
\end{figure}

However there are a couple unique attributes that sets FreeBSD apart from its counterpart. For example if memory is limited for a given system FreeBSD will completely swap a process into swap memory on a hard drive or solid state drive in server cases. However this method of context swapping is not unique to FreeBSD alone. FreeBSD and Linux both utilize similar naming conventions for their malloc and freeing functions like zalloc. Zalloc is normally used in embedded device however they serve as an easy method for mallocing or freeing space with zeros, instead of doing it manually in kernel or user level programs\cite{freeBsdBook}. Where its unique attributes show regarding memory management is when memory is moderately limited to all processes on a system, The FreeBSD Kernel then tunes shared amounts of memory to each process on the system evenly distributing them across physical memory\cite{25Memory88:online}. However exceptions can be made if on initialization or at any point during run time the process tells the kernel how much memory it will need in the future, aiding in performance overall for the system. As a design concept for FreeBSD that has been kept for nearly ten years is memory is relative expensive and small to use for a FreeBSD operating system, while connected disks were large and fast. Thus frugality in memory usage was favored sacrificing extra disk input and output\cite{freeBsdBook}. To combat completely running out of memory for any given process, FreeBSD allows process and threads the ability to share parts of a mapped virtual memory space in Random Access Memory. This results in any change by one process in the kernel to be visible by another process that shares that data, and vice versa, this system was implemented to slowly replace the older system of limiting memory to all processes overall.
\subsection{Abstract}
Each operating system whether it be Windows, FreeBSD, or Linux implements roughly the same concepts of memory management. They all implement some form of memory address indexing in virtual memory space that servers as an in between for physical memory and paging allocations. this management system abstracts the lower level architecture of the kernel and digital logic behind keeping process data. All systems implement a form of paging that holds all quickly accessible data for kernel level processes. And they all implement the same functionality of allocation and reallocation of pages. Where their differences reside in FreeBSDs is the way of handling limited physical memory by limiting resources overall to all processes, and sharing memory addresses to all process. While In Windows it is the sizes of the pages themselves (even down to the naming conventions example is huge, vs large and small). To the structuring of the the tree that helps map addresses to the disk. However where all operating systems (Windows, FreeBSD, and Linux) have some exact similarities is the ability to use the mapping library to more easily index files and contents of data to varying address spaces in the kernel.
