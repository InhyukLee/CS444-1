\documentclass[letterpaper,10pt,titlepage]{IEEEtran}

\usepackage{listings}
\lstdefinestyle{customc}{
  belowcaptionskip=1\baselineskip,
  breaklines=true,
  frame=L,
  xleftmargin=\parindent,
  language=C,
  showstringspaces=false,
  basicstyle=\footnotesize\ttfamily,
  keywordstyle=\cred,
  identifierstyle=\cblue,
  stringstyle=\color{orange},
}

\usepackage{hyperref}% http://ctan.org/pkg/hyperref
\hypersetup{%
  colorlinks = true,
  linkcolor  = black
}

\usepackage{graphicx}                                        
\usepackage{amssymb}                                         
\usepackage{amsmath}                                         
\usepackage{amsthm}   


\usepackage{alltt}                                           
\usepackage{float}
\usepackage{color}
\usepackage{url}

\usepackage{balance}
\usepackage[TABBOTCAP, tight]{subfigure}
\usepackage{enumitem}
\usepackage{pstricks, pst-node}

\usepackage{geometry}
\geometry{textheight=8.5in, textwidth=6in}

\newcommand{\cred}[1]{{\color{red}#1}}
\newcommand{\cblue}[1]{{\color{blue}#1}}

\usepackage{hyperref}
\usepackage{geometry}

\def\name{Kevin Turkington}
\author{\name}
\title{Kernel Constructs and Comparisons\\
  	\large CS444 Final, Spring 2017}


%% The following metadata will show up in the PDF properties
\hypersetup{
  colorlinks = true,
  urlcolor = black,
  pdfauthor = {\name},
  pdfkeywords = {cs444 ``operating systems 2''Final},
  pdfpagemode = UseNone
}

\begin{document}
  \begin{titlepage}
      \maketitle
  \end{titlepage}
  
\section{introduction}
Comparisons and differences will be made in context to Linux, Windows, and FreeBSD's vanilla configurations or any pre-installed option in any of the listed systems. As an overview, in most kernel implementations of drivers, file manipulation, and overall system handling between Linux and FreeBSD are extremely similar. With Windows often being the outlier, taking a different approach to these abstracted concepts. In this essay we will go over implementation of processes, overall I/O, memory management, and interrupts Between these popular operating system choices.

\input{writing1.tex}

\input{writing2.tex}

\section{Memory Management}
\subsection{Overview}
Memory management in Windows, FreeBSD, and Linux is the process of creating addresses that directly reference specific areas of a physical memory. This involves but not limited to RAM (Random Access Memory), memory caches, and flash based memory from devices like SSD's (solid state drives)\cite{Whatisme93:online}. These abstractions of physical memory are normally kept in virtual memory space as stated above in the form of addresses. For Linux this is typically done via pages and paging tables while other operating systems for example windows utilizes a slight different structured approach, leaving the overall semantics of the concept relatively the same across the board. In any of the given operating systems memory management involves the allocation, reallocation of data to specific blocks of memory while organizing it into a digestible way that both a program can quickly access but more importantly a programmer can utilize when creating user and kernel level applications.
\subsection{Memory in Windows}
Within the Windows kernel memory management is slightly different from the Linux kernel underneath the hood. The first major difference being page sizing. In Windows pages can be one of two sizes large (2MB) and small (4KB)\cite{windowsbookpt1}, mean while Linux utilizes (depending on the architecture of the system) 4KB and 8KB page sizes\cite{Linux_book}. These pages for both systems are managed in a similar way concept wise, through the use of paging directories and tables. Address mappings for memory are structured much like a tree branching from the root to different tables depending on the index of a address space to the page itself. This structure minimizes search times for a page and organizes addresses in a uniform way for kernel level developers creating drivers. While pages in Linux are the back bone of data management that can be manipulated by drivers, in Windows pages require several additional services like managers for working sets, process and stack swappers, page writers, and a thread dedicated to writing zeros to a page (because alternative methods give significant performance impacts to the Windows Kernel). Additionally pages in Windows have different states, while Linux pages have a relative first come first serve bases depending on the process allocating memory. These states in the Windows kernel are as specified: free (allocated but unused page), reserved (will be allocated and used, a preemptive approach to memory allocation for a process), committed (private to a specific process, and the page is allocated), and shareable (Non private, mapped and can be accessed by any process or thread). These states for pages in Windows help set priorities for allocation of processes.
\subsection{Memory in FreeBSD}
FreeBSD and Linux's memory management methods are extremely similar because many of its core components are exactly the same.
\begin{figure}[h]
\begin{lstlisting}[language=C, style=customc]
void *mmap(addr, len, prot, flags, fd, offset)
  void		*addr;
  size_t	len;
  int		prot;
  int		flags;
  int		fd;
  off_t		offset;
{
  if (__getosreldate() >= 700051)
    return (__sys_mmap(addr, len, prot, flags, fd, offset));
  else
    return (__sys_freebsd6_mmap(addr, len, prot, flags, fd, 0, offset));
}
\end{lstlisting}
\caption{ Memory mapping in FreeBSD\cite{mmapcinf8:online}}
\end{figure}

However there are a couple unique attributes that sets FreeBSD apart from its counterpart. For example if memory is limited for a given system FreeBSD will completely swap a process into swap memory on a hard drive or solid state drive in server cases. However this method of context swapping is not unique to FreeBSD alone. FreeBSD and Linux both utilize similar naming conventions for their malloc and freeing functions like zalloc. Zalloc is normally used in embedded device however they serve as an easy method for mallocing or freeing space with zeros, instead of doing it manually in kernel or user level programs\cite{freeBsdBook}. Where its unique attributes show regarding memory management is when memory is moderately limited to all processes on a system, The FreeBSD Kernel then tunes shared amounts of memory to each process on the system evenly distributing them across physical memory\cite{25Memory88:online}. However exceptions can be made if on initialization or at any point during run time the process tells the kernel how much memory it will need in the future, aiding in performance overall for the system. As a design concept for FreeBSD that has been kept for nearly ten years is memory is relative expensive and small to use for a FreeBSD operating system, while connected disks were large and fast. Thus frugality in memory usage was favored sacrificing extra disk input and output\cite{freeBsdBook}. To combat completely running out of memory for any given process, FreeBSD allows process and threads the ability to share parts of a mapped virtual memory space in Random Access Memory. This results in any change by one process in the kernel to be visible by another process that shares that data, and vice versa, this system was implemented to slowly replace the older system of limiting memory to all processes overall.
\subsection{Abstract}
Each operating system whether it be Windows, FreeBSD, or Linux implements roughly the same concepts of memory management. They all implement some form of memory address indexing in virtual memory space that servers as an in between for physical memory and paging allocations. this management system abstracts the lower level architecture of the kernel and digital logic behind keeping process data. All systems implement a form of paging that holds all quickly accessible data for kernel level processes. And they all implement the same functionality of allocation and reallocation of pages. Where their differences reside in FreeBSDs is the way of handling limited physical memory by limiting resources overall to all processes, and sharing memory addresses to all process. While In Windows it is the sizes of the pages themselves (even down to the naming conventions example is huge, vs large and small). To the structuring of the the tree that helps map addresses to the disk. However where all operating systems (Windows, FreeBSD, and Linux) have some exact similarities is the ability to use the mapping library to more easily index files and contents of data to varying address spaces in the kernel.


\section{interrupts and synchronization}
\subsection{Overview}
Interrupts are the performance efficient alternative to polling for any operating system level process or user-land program. They offer programmers the ability to notify a CPU at anytime indicating a program/kernel critical change that needs attention and quantum time from the CPU. synchronization within the kernel are essentially an abstraction of interrupts on the kernel and user land levels. Synchronizing allows multiple concurrent processes to share data between one another without potentially corrupting it by altering it mid read or write. These abstractions of interrupts are implemented in the form of mutexes, semaphores, spin locks, and many others. These implementations however vary slightly between operating system to operating system, either by varying syntax with the underlying semantics being relatively the same. With a concise definition of interrupts as well as synchronization have been established, it will help aid in the development of the comparisons that will be made between Windows, FreeBsd, and Linux.
\subsection{Interrupts in Windows}
Interrupts in windows are typically handled by a trap dispatching mechanism. This allows for a driver to force a processor to temporarily suspend what it is currently doing in favor of a newly executing thread. These trap handlers help the processor distinguish if a given interrupt can be run synchronously vs asynchronously, as well as determine the type of interrupt (hardware or software related). Once a interrupt is caught by the trap handler in windows it is sent to be dispatched by Interrupt service routines much like Linux. Additionally interrupts are indexed to the IDT (interrupt dispatch table), much like Linux's system call table it allows programs in user space to create various calls to a function by its interrupt number to retrieve data from the kernel or run various kernel routines. However the difference between the systems is its limitation of two hundred and fifty six entries per processor (logical core) for a windows IDT\cite{windowsbookpt1}. Many interrupts in the Windows system are reliant on synchronization, because they may edit or read data that can be altered by another process at the same time.
\subsection{Synchronization in Windows}
Much like Linux Overall the same concept of mutual exclusion is shared between the operating systems as a crucial element of their development. The key difference is how the concept of mutual exclusion is abstracted. For Windows this is done through the use of High-IRQL's and Low-IRQL's, instead of its Linux counter part of spin-locks and semaphores. High-IRQL's contain all synchronization mechanisms for any kernel level process that requires absolute mutual exclusion of a data structure. This is done by masking the level of kernel level process using the High-IRQL, preventing other kernel level processes from overriding control (this is similar to Linux's spin-lock synchronization). Low-IRQL's are the Linux equivalent of mutexes and semaphores, they allows user level threads to gain exclusive access to non critical data, this form of IRQL does require the one of the following locking primitives: condition variables, slim reader-writer locks, run-once initialization, or critical sections, to execute an interrupt.

\begin{figure}[h]
\begin{lstlisting}[language=C, style=customc]
NTSTATUS DispatchRead(
  __in struct _DEVICE_OBJECT  *DeviceObject,
  __in struct _IRP  *Irp
  )
{
  // this will be called at irql == PASSIVE_LEVEL
  ...
  // we have acquire a spinlock
  KSSPIN_LOCK lck;
  KeInititializeSpinLock( &lck );
  KIRQL prev_irql;
  KeAcquireSpinLock( &lck,&prev_irql );
  
  // KeGetCurrentIrql() == DISPATCH_LEVEL 

  KeReleaseSpinLock( &lck, prev_irql );
  // KeGetCurrentIrql() == PASSIVE_LEVEL 
  ...
}
\end{lstlisting}
\caption{ Generic device driver in Windows\cite{windowsD95:online}}
\end{figure}

This works differently that High-IRQL's that they don't interrupt the processors in any way, and are restricted to user-level programs only\cite{windowsbookpt1}.
\subsection{Interrupts in FreeBSD}
In FreeBSD interrupts are handled by two different methods system calls, or the more traditional trap handlers. Trap handlers normally occur because of "unintentional errors, such as division by zero or indirection through an invalid pointer"\cite{freeBsdBook}. When a trap handler catches an interrupt it starts by saving the current state of the erroring process, determines interrupt information, checks higher-priority processes if they are affected, and exits safely. However when interrupts are created from system calls instead of kernel level errors in the system they can block for data causing mutual exclusion within the kernel. In the FreeBSD Kernel software interrupts have the highest priority, and are typically used for network devices because of their random retrieval of packets\cite{freeBsdBook}, while hardware level interrupts are serviced after the fact. When any interrupt is serviced they are referred to as heavyweights, because a full service switch is require by the processor in order to begin assigning quantums\cite{83Genera16:online}. 
\subsection{Synchronization in FreeBSD}
However there are options to the programmer for quicker lighter weight interrupts using the INTR\_FAST flag, this flag was mainly used in earlier version of the FreeBSD kernel, and generally unused because its requires additional used memory and processor time\cite{83Genera16:online}. Synchronization in FreeBSD unlike windows contains many of the same naming schemes as its Linux counterpart, however many of the mechanics it abstracts are very different. For example in user space programs, programmers have the option to use mutexes to ensure only one thread at a time can access data. Within FreeBSD there are rules set to prevent deadlocking of processes, to achieve this threads with immediate access to multiple pools of data may only lock one mutex at a time\cite{43Contex5:online}. This fail safe however is not implemented by the kernel in Linux, therefore requiring elegant solutions like those in the dining philosophers problem.
\subsection{Abstract}
Implementation of interrupts and synchronization between Windows, FreeBSD, and Linux generally use similar constructs. With the only major different between them being implementation of varying locks, and method of creating interrupts themselves. For Windows its biggest difference is the use of IRQs and how trap handlers relay information about a given interrupt. For FreeBSD it is mainly trap handler memory and CPU usages, otherwise in comparison to Linux they show similar traits.\\


\section{conclusion}
In the preceding topics, I've provided an overview of Kernel implementations of various drivers, system calls, and other constructs in the context of Windows, and FreeBSD in comparison to Linux. As a overview observation of each of the kernel related functionality, there is a theme of FreeBSD and Linux containing many extreme similarities, with Windows taking a different route in terms core concepts implementation. However Windows, FreeBSD, and Linux keep the overall semantics of processes, I/O, memory, and Interrupts on an abstracted view the same.

% bibliography
\nocite{*}%if nothing is referenced it will still show up in refs
\bibliographystyle{plain}
\bibliography{refs}
%end bibliography

\end{document}
