\documentclass[letterpaper,10pt,titlepage]{IEEEtran}

\usepackage{graphicx}                                        
\usepackage{amssymb}                                         
\usepackage{amsmath}                                         
\usepackage{amsthm}                                          

\usepackage{alltt}                                           
\usepackage{float}
\usepackage{color}
\usepackage{url}

\usepackage{balance}
\usepackage[TABBOTCAP, tight]{subfigure}
\usepackage{enumitem}
\usepackage{pstricks, pst-node}

\usepackage{geometry}
\geometry{textheight=8.5in, textwidth=6in}

\newcommand{\cred}[1]{{\color{red}#1}}
\newcommand{\cblue}[1]{{\color{blue}#1}}

\usepackage{hyperref}
\usepackage{geometry}

\def\name{Alessandro Lim, Kevin Turkington}
\author{\name}
\title{CS 444 Assignment 4}

%pull in the necessary preamble matter for pygments output
%\input{pygments.tex}

%% The following metadata will show up in the PDF properties
\hypersetup{
  colorlinks = true,
  urlcolor = black,
  pdfauthor = {\name},
  pdfkeywords = {cs444 ``operating systems 2'' assignment 4},
  pdftitle = {CS 444 Assignment 4},
  pdfpagemode = UseNone
}

\begin{document}
\maketitle
\hrulefill

\section{Assignment4 Questions}
\subsection{What do you think the main point of this assignment is?}
The main point of this assignment was to familiarize our selfs with the kernel, and gain the ability to find resources to aid with testing. And familiarizing our selfs with different algorithms and the various ways they can be implemented into the kernel.

\subsection{How did you personally approach the problem? Design decisions, algorithm, etc.} 
First we found pseudo code for the best fit algorithm, and tried to match our kernel implementation as closely as possible to the found algorithm.

\subsection{How did you ensure your solution was correct? Testing details, for instance.} 
We knew our implementation was correct when the kernel didn't hang on boot forever, after several attempt creating the best fit slob. The we created a system call to get fragmentation rates of the disk, and on average the best fit algorithm did better than first fit.

\subsection{What did you learn?}
forgetting a single line of code can lead to alot of headaches. and there are many ways to verify if an implementation is running, its just a matter of finding which way works best.

\begin{table*}
\section{Work Log}
\begin{tabular}{l l c l}\textbf{Date} & \textbf{Name} & \textbf{Hours} & \textbf{Description}\\\hline
5/10 & Kevin & 3 & sbd and crypto API research\\\hline
5/11 & Kevin & 3 & sbd and crypto API research\\\hline
5/12 & Kevin & 1 & sbd and crypto API research\\\hline
5/14 & Kevin & 5 & sbd driver fixing\\\hline
5/15 & Kevin & 3 & sbd driver encryption\\\hline
5/16 & Kevin & 5 & sbd driver encryption\\\hline
5/16 & Alessandro & 1 & sbd and crypto API research\\\hline
5/16 & Alessandro & 1 & concurrency research\\\hline
5/16 & Alessandro & 1 & starting concurrency3\\\hline
5/17 & Alessandro and Kevin & 3 & Concurrency3\\\hline
5/18 & Alessandro and Kevin & 1 & Finished
concurrency3\\\hline
5/18 & Kevin &5 & sbd\_enc.c\\\hline
5/18 & Kevin &5 & cleaning up sbd\_enc.c and demoscript\\\hline
5/22 & Kevin &2 & Writeup\\\hline
\end{tabular}
\end{table*}

\begin{table*}
\section{Gitlog}
\begin{tabular}{l l l}\textbf{Detail} & \textbf{Author} & \textbf{Description}\\\hline
\href{https://github.com/zainkai/CS444/commit/3d519b7548c1b98a8a863850ae6e0153ab224225}{3d519b7} & Zainkai & concurrency 4 needs verificarion too tired\\\hline
\href{https://github.com/zainkai/CS444/commit/359910dddcf2fc23681d384d230264c7abf1eac2}{359910d} & Zainkai & more in sync\\\hline
\href{https://github.com/zainkai/CS444/commit/1323ecf1dc0b46e2635b638972b261c53e83e64f}{1323ecf} & Zainkai & sleeping barber final\\\hline
\href{https://github.com/zainkai/CS444/commit/68a2aec0b04688107dfcda4da28b9bbdb1ae8511}{68a2aec} & Zainkai & defaults for assignment 4\\\hline
\href{https://github.com/zainkai/CS444/commit/06f5b6787b8c183fa292032a320d1beb8e0a108c}{06f5b67} & Zainkai & rm slob\\\hline
\href{https://github.com/zainkai/CS444/commit/90799ca9b5a5fff6c781ef46c55c3d1c297a2ebf}{90799ca} & Zainkai & FINAL THIS TIME I SWEAR SLEEPING BARBER\\\hline
\href{https://github.com/zainkai/CS444/commit/05b57056c858d1c359f1725895d908939d7a5c75}{05b5705} & Kevin Turkington & merge\\\hline
\href{https://github.com/zainkai/CS444/commit/9ae713cdc5db0f6f9a41329259036811bafd841d}{9ae713c} & Kevin Turkington & merge\\\hline
\href{https://github.com/zainkai/CS444/commit/0776f2dd612cd32879617c567eb4d065295a281b}{0776f2d} & Kevin Turkington & deleting program file\\\hline
\href{https://github.com/zainkai/CS444/commit/6e2a1db8ef830076841f23744e468a6b54947335}{6e2a1db} & Kevin Turkington & clean up\\\hline
\href{https://github.com/zainkai/CS444/commit/b43ce0ee0eb5df33ff82de8e2f9ed22beac7d548}{b43ce0e} & Kevin Turkington & adding reasource\\\hline
\href{https://github.com/zainkai/CS444/commit/b68d2374846d9560017866c5eaeb9eaf95d4b041}{b68d237} & Zainkai & final iterating concurrency\\\hline
\href{https://github.com/zainkai/CS444/commit/f3d34a152be1e494673e359f41161551725558a4}{f3d34a1} & Zainkai & merging\\\hline
\href{https://github.com/zainkai/CS444/commit/7059979332b746cf786760659a10bd88a97d6b1d}{7059979} & Kevin Turkington & merge\\\hline
\href{https://github.com/zainkai/CS444/commit/c497cad1a1cd1a7000c86bb569e9379454e2219f}{c497cad} & Kevin Turkington & merge\\\hline
\href{https://github.com/zainkai/CS444/commit/a9d2f1ac4339c8417c4417ee72b80d330284dd79}{a9d2f1a} & alessanf & Merge pull request \#1 from zainkai/master\\\hline
\href{https://github.com/zainkai/CS444/commit/19d179c85a67aca08d0251dbeab418b726df026d}{19d179c} & Kevin Turkington & im confident this will compile first go\\\hline
\href{https://github.com/zainkai/CS444/commit/4183453f70d8da8e11ab86a98cb45af6f9516612}{4183453} & Kevin Turkington & adding edited by\\\hline
\href{https://github.com/zainkai/CS444/commit/59af7807ac70cfc988412f0cca6dddd7ec843943}{59af780} & Zainkai & refactor 1\\\hline
\href{https://github.com/zainkai/CS444/commit/53b984f23b8bc3ee20f48395f3d2cc5fda38c600}{53b984f} & Zainkai & finally a working version\\\hline
\href{https://github.com/zainkai/CS444/commit/9c3f87a7966b54fc340ae3981fcde399089ce65a}{9c3f87a} & Zainkai & v1.1\\\hline
\href{https://github.com/zainkai/CS444/commit/5fb3a96d3ef4e47a9de698e32d78ba0ba9f45a7b}{5fb3a96} & Zainkai & deleteing foler\\\hline
\href{https://github.com/zainkai/CS444/commit/b0d22145ac25d1377ae8500f1d803d3f7be6388c}{b0d2214} & alessanf & Merge remote-tracking branch 'refs/remotes/zainkai/master'\\\hline
\href{https://github.com/zainkai/CS444/commit/8977403e52cc6707259e21c0027d3f6171ac322c}{8977403} & Zainkai & comparision tests for best and first fit slobs\\\hline
\href{https://github.com/zainkai/CS444/commit/0c0efb943c849d62ece32ba712997fbad2cae1c3}{0c0efb9} & Kevin Turkington & TODO patch files and demoscript\\\hline
\href{https://github.com/zainkai/CS444/commit/b8cf601f769f21b74795e1d1e237feff5db60f59}{b8cf601} & Zainkai & editing fragmentation program\\\hline
\href{https://github.com/zainkai/CS444/commit/73fbb1063a9aa805bf213fc7d33642c1e6e2a9ff}{73fbb10} & Zainkai & Merge branch 'master' of https://github.com/zainkai/CS444\\\hline
\href{https://github.com/zainkai/CS444/commit/9a19a49ae47d00125e8effdce9ac3cba4f13def5}{9a19a49} & alessanf & Merge pull request \#5 from alessanf/master\\\hline
\href{https://github.com/zainkai/CS444/commit/c78070fcfcf3a16ec999190abb19edfcee4fef0c}{c78070f} & alessanf & Merge remote-tracking branch 'refs/remotes/zainkai/master'\\\hline
\href{https://github.com/zainkai/CS444/commit/8e569f56441fbe67860382e4cd39816c39dcdc40}{8e569f5} & alessanf & Merge branch 'master' of https://github.com/alessanf/CS444-1\\\hline
\href{https://github.com/zainkai/CS444/commit/cd81b86b888c4b4ac37193c7036e8fbab3a6b855}{cd81b86} & alessanf & added part1\\\hline
\href{https://github.com/zainkai/CS444/commit/11c1018c9cbbd0db3668ad5e5f8e8de8a14cc8e2}{11c1018} & alessanf & part1 is bugged, fixing it\\\hline
\href{https://github.com/zainkai/CS444/commit/1d2f492c2d2634a67881b6a91c78c38a8c4fcf3b}{1d2f492} & alessanf & part 1 is done\\\hline
\href{https://github.com/zainkai/CS444/commit/aeb7a2492835ea2e0ce3fd8afd81be47c140c7cd}{aeb7a24} & alessanf & changes to smoker\\\hline
\href{https://github.com/zainkai/CS444/commit/a775c84bbe228a41210798c53400395878a77588}{a775c84} & alessanf & smoker problem done\\\hline
\href{https://github.com/zainkai/CS444/commit/f90921414a60c2807a15dd69599080a310d91552}{f909214} & alessanf & Merge remote-tracking branch 'refs/remotes/origin/master' into zainkai/master\\\hline
\href{https://github.com/zainkai/CS444/commit/3c5738afb23b99bb452ffba8dea432bb81bfaa66}{3c5738a} & alessanf & Merge remote-tracking branch 'refs/remotes/origin/master' into zainkai/master\\\hline
\href{https://github.com/zainkai/CS444/commit/e5da61207fb0c228a55aa2af1f8e1b12c9b20b71}{e5da612} & alessanf & Merge pull request \#6 from alessanf/master\\\hline
\href{https://github.com/zainkai/CS444/commit/8fa3fd8f2913a4036814808598ab87ae0bf9ded7}{8fa3fd8} & Zainkai & writing3\\\hline\end{tabular}

\end{table*}

    
% bibliography
\nocite{*}%if nothing is referenced it will still show up in refs
\bibliographystyle{plain}
\bibliography{refs}
%end bibliography

%input the pygmentized output of mt19937ar.c, using a (hopefully) unique name
%this file only exists at compile time. Feel free to change that.
\end{document}
